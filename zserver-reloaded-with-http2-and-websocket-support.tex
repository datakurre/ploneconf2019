\documentclass[12pt,aspectratio=169]{beamer}

% Input
\usepackage[utf8]{inputenc}
\usepackage[T1]{fontenc}
\usepackage[letterspace=100]{microtype}
\usepackage{upquote}

% Beamer
\usetheme{metropolis}
\usepackage[sfdefault]{FiraSans}
\usepackage{xcolor}
\definecolor{blue}{HTML}{002957}
\definecolor{red}{HTML}{F1563F}
\setbeamercolor{palette primary}{fg=white,bg=blue}
\setbeamercolor{title separator}{fg=red,bg=blue}
\setbeamercolor{frametitle}{fg=white,bg=blue}
\setbeamercolor{progress bar}{fg=red,bg=blue}
\setbeamercolor{alerted text}{fg=red,bg=blue}
\makeatletter
\setlength{\metropolis@titleseparator@linewidth}{2pt}
\setlength{\metropolis@progressonsectionpage@linewidth}{2pt}
\setlength{\metropolis@progressinheadfoot@linewidth}{2pt}
\makeatother

% Graphics
\usepackage{graphicx}
\usepackage{epstopdf}
\DeclareGraphicsExtensions{.png,.pdf,.eps}
\tikzset{
  invisible/.style={opacity=0},
  visible on/.style={alt={#1{}{invisible}}},
  alt/.code args={<#1>#2#3}{%
    \alt<#1>{\pgfkeysalso{#2}}{\pgfkeysalso{#3}} % \pgfkeysalso doesn't change the path
  },
}
\usetikzlibrary{arrows, shapes}

% Various
\usepackage{hyperref}
\usepackage{minted}
\usepackage{fontawesome}

% Title
\title{ZServer Reloaded with HTTP/2 and WebSocket Support}
\subtitle{Plone Conference 2019}
\date{25.10.2019}
\author{Asko Soukka}
\institute{\vspace{-1.5cm}\includegraphics[height=1.5cm]{images/jyu-vaaka-kaksikielinen.eps}\hfill\includegraphics[width=0.20\paperwidth]{images/plone-icon.pdf}\raisebox{0.075\paperwidth}}

\newcommand{\setmytemplate}{\usebackgroundtemplate{\begin{picture}(0,0)(-5,250)\footnotesize\href{https://twitter.com/datakurre}{\faTwitter~datakurre}\end{picture}\begin{picture}(0,0)(-330,298)\includegraphics[width=0.40\paperwidth]{images/plone-icon.pdf}\end{picture}}}

\begin{document}

% define tikz styles
\definecolor{gatsby}{HTML}{663399}
\tikzstyle{block} = [rectangle, fill=gatsby, text=white,
    text width=5em, text centered, rounded corners, minimum height=4em]
\tikzstyle{example} = [rectangle, draw, dashed,
    text width=5em, text centered, rounded corners, minimum height=4em]
\tikzstyle{line} = [draw, -latex']

\setbeamertemplate{background canvas}[default]
\maketitle

%---------------------------------------------------------------------------------------

\setbeamertemplate{background canvas}[default]
\section{Demo}

%---------------------------------------------------------------------------------------

\setbeamertemplate{background canvas}[default]
\section{Status}

%---------------------------------------------------------------------------------------

\setmytemplate
\begin{frame}{Drop-in Replacement}
  \textbf{HTTP and WebDAV}
  \begin{itemize}
    \item Twisted HTTP Server
    \item Twisted thread pool w/ async FileSender
    \item AsyncIO/uvloop loop when available
    \item ZConfig configuration and logging
    \item WSGI ZPublisher request handler
  \end{itemize}
  \textbf{HTTP/2}
  \begin{itemize}
    \item Twisted Web OpenSSL + H2
  \end{itemize}
\end{frame}

%---------------------------------------------------------------------------------------

\setmytemplate
\begin{frame}{WebSockets}
  \textbf{WebSockets with PubSub-messaging}
  \begin{itemize}
    \item Autobahn Twisted WebSocket protocol
    \item ZMQ PubSub-socket using IPC-socket
    \item Connection upgradeable at all paths
    \item Connection autosubscribes to local events
  \end{itemize}
\end{frame}

%---------------------------------------------------------------------------------------

\setmytemplate
\begin{frame}[fragile]{Publish-Subscribe}
  \textbf{Client subscribes to events}
  \footnotesize
  \begin{minted}{JSON}
{"method": "SUBSCRIBE": "path": "/"}
  \end{minted}
  \normalsize
  \textbf{Client unsubscribes events}
  \footnotesize
  \begin{minted}{JSON}
{"method": "UNSUBSCRIBE": "path": "/"}
  \end{minted}
\end{frame}

\begin{frame}[fragile]{Guarded Events}
  \textbf{Server publishes guarded event}
  \footnotesize
  \begin{minted}{JSON}
{
  "guards": [{
    "allowedRolesAndUsers": {
      "method": "/Plone/@@wsevents-guard",
      "tokens": ["Editor"],
    },
  }],
  "payload": {}
}
  \end{minted}
\end{frame}

\begin{frame}[fragile]{Guard Response}
  \textbf{Server responses to guard check once for each subscriber}
  \footnotesize
  \begin{minted}{JSON}
{
  "allowedRolesAndUsers": {
    "tokens": ["Reviewer", "Editor"],
    "expires": 1571723078
  }
}
  \end{minted}
\end{frame}

%---------------------------------------------------------------------------------------

\setmytemplate
\begin{frame}{State of Things}
  \textbf{\href{https://github.com/datakurre/ZServer/branches/active}{github.com/datakurre/ZServer}}
  \begin{itemize}
    \item Rebased on top of the latest ZServer release
    \item Python 3 syntax update + new code
    \item Used in GSOC 2019 for \texttt{gatsby-source-plone}
    \item WebDAV tested manually at PLOG 2019
    \item Dead code not removed yet and tests are failing\dots
  \end{itemize}
\end{frame}

%---------------------------------------------------------------------------------------

\setbeamertemplate{background canvas}[default]
\begin{frame}[fragile]{State of Friends}
  \textbf{Complete experience}
  \footnotesize
  \begin{minted}{cfg}
[sources]
ZServer = git git@github.com:datakurre/ZServer branch=datakurre/master
plone.recipe.zope2instance = git git@github.com:plone/plone.recipe.zope2instance
                             branch=datakurre/zserver
collective.wsevents = git git@github.com:datakurre/collective.wsevents
plonectl = git git@github.com:datakurre/plonectl
collective.taskueue = git git@github.com:collective/collective.taskqueue
                      branch=python3_compatibility
  \end{minted}
\end{frame}

%---------------------------------------------------------------------------------------

\setbeamertemplate{background canvas}[default]
\section{Next}

%---------------------------------------------------------------------------------------

\setmytemplate
\begin{frame}{State of Upstream}
  \textbf{Zope 4}
  \begin{itemize}
    \item WebDAV made dependent on ZServer (by mistake?)
  \end{itemize}
  \textbf{Zope 5}
  \begin{itemize}
    \item Assumes ZServer (and WebDAV?) to no longer exist
  \end{itemize}
\end{frame}

%---------------------------------------------------------------------------------------

\setmytemplate
\begin{frame}{Discussion}
  \textbf{Remaining steps}
  \begin{itemize}
    \item Upstream or not?
    \item Eliminate dead code
    \item QA (tests, documentation, \dots)
    \item Release
    \item Restore \texttt{wsgi=off} on Python 3
  \end{itemize}
\end{frame}

%---------------------------------------------------------------------------------------

\setbeamertemplate{background canvas}[default]
\begin{frame}[standout]
\vfill
\includegraphics[height=0.50\paperheight]{images/plone-icon-hearts.png}
\par
\href{https://datakurre.github.io/ploneconf2019/zserver}{datakurre.github.io/ploneconf2019/zserver}
\end{frame}

%---------------------------------------------------------------------------------------

\end{document}
